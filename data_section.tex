% hello.tex - my first LaTeX for TZA project example

\documentclass{article}

\begin{document}
\title{Optimal fertilizer use on maize plots in Tanzania: Data}
\maketitle

\section{winsoring data}

1. restrict to a range that is feasible
2. winsor values.

The data available from the world bank contains outliers that needed to be removed for accurate analysis. Some literature has trimmed and some has used winsoring to deal with outliers. Either approach has it's limitations. In this paper we take a conservative approach by first limiting the data to a likely or plausible range, thereby removing the most extreme of outliers. Further data cleaning proceeds by observing plots and testing whether such large values are likely. Less than ?% of data points were removed.

An important point here is that the small size of size plots measured in hectacres results in large values of yield which bias the results upwards. As we are interested primarily in small holder farmers we restruict our attention to farmers who have plots less than 10 hectacres and greater than 0.01 hectacres. Given that the GPS measurements had been collected by GPS equipment it did not seem to appropriate to question these variables.

Liverpool-Tasie use 700 kg/ha as a cut off point, however, as we have only a single value between 300 and 700 we get rid of anything greater than 700. Compared to our original simple we lose only ? plots. 

A further indication of the need to remove erroneous fertilizer values is that the maximum nitrogen recored in 2008 was around 500 kg/ha. This indicates that there is an issue with the fertilizer values.

% Table of key statitsics for 2010 farmers and plots by zone
\begin{table}
	\caption{summary statistics for 2010 by zone}
	\label{}
\begin{tabular}{@{\extracolsep{5pt}}lcccccc}
\\[-1.8ex]\hline 
\hline \\[-1.8ex]
Zone & \multicolumn{1}{c}{No. Farmers} & \multicolumn{1}{c}{No. plots} & \multicolumn{1}{c}{Avg Yield} & \multicolumn{1}{c}{Avg Nitrogen} & \multicolumn{1}{c}{Farmers using Nitrogen} &\multicolumn{1}{c}{Plots with Nitrogen} \\
\hline
Central & 94 & 128 & 797.53 & 0.81 & 4.3 & 3.9 \\
Eastern & 71 & 98 & 967.80 & 0.55 & 4.2 & 3.1 \\
Lake & 85 & 102 & 768.20 & 0.00 & 0.0 & 0.0 \\
Northern & 207 & 265 & 1329.55 & 5.00 & 12.6 & 10.2 \\
Southern & 258 & 356 & 806.89 & 9.73 & 17.8 & 19.4 \\
Southern Highlands & 276 & 507 & 1336.06 & 19.52 & 43.8 & 40.2 \\
Western & 140 & 199 & 701.92 & 4.31 & 12.1 & 10.6 \\
Zanzibar & 10 & 13 & 416.19 & 0.00 & 0.0 & 0.0 \\
\hline
\end{tabular}
\end{table}

% table of summary statistics for 2010 farmers and plots by region
\begin{table}
	\caption{summary statitsics for 2010 by region}
	\label{}
\begin{tabular}{@{\extracolsep{5pt}}lcccccc}
\\[-1.8ex]\hline 
\hline \\[-1.8ex]
Region & \multicolumn{1}{c}{No. Farmers} & \multicolumn{1}{c}{No. plots} & \multicolumn{1}{c}{Avg Yield} & \multicolumn{1}{c}{Avg Nitrogen} & \multicolumn{1}{c}{Farmers using Nitrogen} &\multicolumn{1}{c}{Plots with Nitrogen} \\
\hline
Arusha & 42 & 51 & 1397.42 & 6.60 & 14.3 & 11.8 \\ 
Dodoma & 54  & 76 & 851.69 & 0.00 & 0.0 & 0.0 \\
Iringa & 95 & 177 & 1398.81 & 31.88 & 68.4 & 62.7 \\
Kagera & 19 & 23 & 849.24 & 0.00 & 0.0 & 0.0 \\
KASKAZINI PEMBA & 8 & 11 & 335.26 & 0.00 & 0.0 & 0.0 \\
KASKAZINI UNGUJA & 1 & 1 & 950.40 & 0.00 & 0.0 & 0.0 \\
Kigoma & 19 & 30 & 550.54 & 1.71 & 5.3 & 3.3 \\
Kilimanjaro & 51 & 60 & 1482.98 & 14.57 & 35.3 & 31.7 \\
KUSINI PEMBA & 1 & 1 & 772.20 & 0.00 & 0.0 & 0.0 \\
Lindi & 75 & 87 & 705.77 & 0.00 & 0.0 & 0.0 \\
Manyara & 36 & 47 & 1572.97 & 0.00 & 0.0 & 0.0 \\
Mara & 20 & 26 & 594.87 & 0.00 & 0.0 & 0.0 \\
Mbeya & 115 & 232 & 1426.80 & 17.47 & 41.7 & 34.9 \\
Morogoro & 71 & 98 & 967.80 & 0.55 & 4.2 & 3.1 \\
Mtwara & 92 & 133 & 609.00 & 0.85 & 7.6 & 5.3 \\
Mwanza & 46 & 53 & 818.06 & 0.00 & 0.0 & 0.0 \\
Rukwa & 66 & 98 & 1007.91 & 2.07 & 12.1 & 12.2 \\
Ruvuma & 91 & 136 & 1065.11 & 24.63 & 42.9 & 45.6 \\
Shinyanga & 38 & 54 & 900.03 & 0.03 & 2.6 & 1.9 \\
Singida & 40 & 52 & 718.37 & 1.99 & 10.0 & 9.6 \\
Tabora & 83 & 115 & 648.38 & 7.00 & 18.1 & 16.5 \\
Tanga & 78 & 107 & 1104.23 & 1.08 & 2.6 & 1.9 \\
\hline
\end{tabular}
\end{table}


% Table of key statitsics for 2008 farmers and plots by zone
\begin{table}
	\caption{summary statistics for 2008 by zone}
	\label{}
\begin{tabular}{@{\extracolsep{5pt}}lcccccc}
\\[-1.8ex]\hline 
\hline \\[-1.8ex]
Zone & \multicolumn{1}{c}{No. Farmers} & \multicolumn{1}{c}{No. plots} & \multicolumn{1}{c}{Avg Yield} & \multicolumn{1}{c}{Avg Nitrogen} & \multicolumn{1}{c}{Farmers using Nitrogen} &\multicolumn{1}{c}{Plots with Nitrogen} \\
\hline
Central & 84 & 106 & 942.7 & 54.63 & 7.1 & 5.7 \\
Eastern & 79 & 108 & 552.54 & 6.36 & 2.5 & 1.9 \\
Lake & 64 & 79 & 692.44 & 0.38 & 1.6 & 1.3 \\
Northern & 173 & 221 & 805.29 & 43.36 & 7.5 & 6.8 \\
Southern & 218 & 290 & 680.78 & 40.65 & 18.3 & 19 \\
Southern Highlands & 262 & 433 & 1272.26 & 44.45 & 34 & 30.3 \\
Western & 93 & 115 & 925.64 & 21.41 & 4.3 & 3.5 \\
\hline
\end{tabular}
\end{table}




% Table of summary statistics for 2008 farmers and plots by region
\begin{table}
	\caption{summary statitsics for 2008 by region}
	\label{}
\begin{tabular}{@{\extracolsep{5pt}}lcccccc}
\\[-1.8ex]\hline 
\hline \\[-1.8ex]
Region & \multicolumn{1}{c}{No. Farmers} & \multicolumn{1}{c}{No. plots} & \multicolumn{1}{c}{Avg Yield} & \multicolumn{1}{c}{Avg Nitrogen} & \multicolumn{1}{c}{Farmers using Nitrogen} &\multicolumn{1}{c}{Plots with Nitrogen} \\
\hline
Arusha & 32 & 37 & 1058.43 & 72.75 & 6.2 & 5.4 \\
Dodoma & 59 & 77 & 928.37 & 0 & 0 & 0 \\
Iringa & 95 & 152 & 1220.58 & 49.37 & 50.5 & 48.7 \\
Kagera & 21 & 23 & 776.87 & 0 & 0 & 0 \\
Kigoma & 15 & 20 & 567.03 & 56.83 & 6.7 & 5 \\
Kilimanjaro & 36 & 49 & 900.35 & 38.84 & 30.6 & 26.5 \\
Lindi & 48 & 52 & 551.44 & 0 & 0 & 0 \\
Manyara & 32 & 36 & 995.03 & 0 & 0 & 0 \\
Mara & 15 & 17 & 529.14 & 0 & 0 & 0 \\
Mbeya & 110 & 197 & 1318.75 & 39.42 & 31.8 & 25.9 \\
Morogoro & 65 & 88 & 628.9 & 9.09 & 1.5 & 1.1 \\
Mtwara & 75 & 100 & 616.77 & 33.69 & 9.3 & 7 \\
Mwanza & 28 & 39 & 713.83 & 0.38 & 3.6 & 2.6 \\
Pwani & 14 & 20 & 216.54 & 3.63 & 7.1 & 5 \\
Rukwa & 57 & 84 & 1256.77 & 26.51 & 10.5 & 7.1 \\
Ruvuma & 95 & 138 & 775.91 & 41.66 & 34.7 & 34.8 \\
Shinyanga & 31 & 40 & 824.39 & 7.29 & 6.5 & 5 \\
Singida & 25 & 29 & 980.74 & 54.63 & 24 & 20.7 \\
Tabora & 47 & 55 & 1129.67 & 14.21 & 2.1 & 1.8 \\
Tanga & 73 & 99 & 594.65 & 0 & 0 & 0 \\
\hline
\end{tabular}
\end{table}


% Table created by stargazer v.5.1 by Marek Hlavac, Harvard University. E-mail: hlavac at fas.harvard.edu
% Date and time: di, feb 10, 2015 - 11:46:11
\begin{table}[!htbp] \centering 
  \caption{} 
  \label{} 
\begin{tabular}{@{\extracolsep{5pt}}lccccc} 
\\[-1.8ex]\hline 
\hline \\[-1.8ex] 
Variable & \multicolumn{1}{c}{N} & \multicolumn{1}{c}{Mean} & \multicolumn{1}{c}{St. Dev.} & \multicolumn{1}{c}{Min} & \multicolumn{1}{c}{Max} \\ 
\hline \\[-1.8ex] 
output (kg/h) & 1,686 & 1,105.641 & 1,233.021 & 1.475 & 7,233.074 \\ 
nitrogen (kg/h) & 1,686 & 9.596 & 26.168 & 0.000 & 149.130 \\ 
\hline \\[-1.8ex] 
\end{tabular} 
\end{table} 

% Table created by stargazer v.5.1 by Marek Hlavac, Harvard University. E-mail: hlavac at fas.harvard.edu
% Date and time: di, feb 10, 2015 - 11:55:56
\begin{table}[!htbp] \centering 
  \caption{table of ....} 
  \label{} 
\begin{tabular}{@{\extracolsep{5pt}}lccccc} 
\\[-1.8ex]\hline 
\hline \\[-1.8ex] 
Statistic & \multicolumn{1}{c}{N} & \multicolumn{1}{c}{Mean} & \multicolumn{1}{c}{St. Dev.} & \multicolumn{1}{c}{Min} & \multicolumn{1}{c}{Max} \\ 
\hline \\[-1.8ex] 
output.kgh.new & 1,686 & 1,105.641 & 1,233.021 & 1.475 & 7,233.074 \\ 
nitrogen.kgh & 1,686 & 9.596 & 26.168 & 0.000 & 149.130 \\ 
tot.cap.shh & 1,675 & 957,546.300 & 3,351,420.000 & 265.214 & 26,588,612.000 \\ 
fam.lab.daysh & 1,686 & 160.319 & 235.743 & 0.000 & 1,708.732 \\ 
hir.lab.daysh & 1,686 & 160.319 & 235.743 & 0.000 & 1,708.732 \\ 
\hline \\[-1.8ex] 
\end{tabular} 
\end{table} 





\end{document}